% !TeX root = paper.tex


\begin{center}
		{\gtfamily \Large 概要}
\end{center}
\vspace{1cm}
\thispagestyle{empty}	
%%%%%%%%%%%%%%%%%%%%%%%%%%%%%%% 概要ここから



 この研究の目的は,Webアプリケーションの処理の高速化の手段として,近年注目を集めているWebAssembly(以後,WASMと記載する)というWebアプリケーション上で動作するバイナリフォーマットプログラム(またはそのプログラムを保存したファイル)を用いて,Webグラフィクスレンダリングを高速化可能かどうか,実際にどれほどの高速化が可能なのかを検証することである.
 
 背景として,Webブラウザ上に組み込みで搭載されているAPIを用いて,多様な処理を行えるようになった近年において,Webアプリケーションでも,VR,ARを用いた没入型のアプリケーションが開発できるようになった.その際,なるべくリアルタイムな没入体験をユーザへ提供するにあたって,画面上にグラフィクスをレンダリングする処理の速度が重要になると考えた,これまでそれらのようなコンテンツを作成するにあたって,開発を円滑に行うためのライブラリ等は開発され,提供されてきた.しかし,それらのライブラリよりも高速にどうささせる方法として,WASMを通してWebグラフィクスを扱うことで,高速化を図ることとした.
 
 手法として,これまで,JavaScript(以後,JSと記載)のプログラム言語で記述され,実行可能であったプログラムを,Rustというプログラミング言語を用いて,記述しなおした上で,WASMとして利用できるようにコンパイルし,WASMをJS側からロードして,WASMでコンパイルされた関数をコールさせるという手法である.
 
 この手法が有効であると考えられる理由として,WASMを有効活用できるとされる処理は,多くの数値演算を用いる処理であることとされている.グラフィクスレンダリングにおいて,動きをつけるようなものは,内部的に大量の行列演算を行うこととなる.そのような演算タスクをもつ処理であるならば,WASMの高速な数値演算処理の恩恵を受けることが可能であると考えたためである.
 
 ○○○○○○○○○○○○○○○○○○○○○○○○○○○○○○○○○○○○○○○○○○○○○○○○○○○○ ○○○○ ○○○○○○○○○○○○○○○○○○○○○○○○○○○○○○○○○○○○○○○○○○○○○○○○○○○○○○○○ ○○○○○○○○○○○○○○○○○○○○○○○○○○○○○○○○○○○○○○○○○○○○○○○○○○○○○○○○ ○○○○○○○○○○○○○○○○○○○○○○○○○○○○○○○○○○○○○○○○○○○○○○○○○○○○○○○○ ○○○○○○○○○○○○○○○○○○○○○○○○○○○○○○○○○○○○○○○○○○○○○○○○○○○○○○○○ ○○○○○○○○○○○○○○○○○○○○○○○○○○○○○○○○○○○○○○○○○○○○○○○○○○○○○○○○ ○○○○○○○○○○○○○○○○○○○○○○○○○○○○○○○○○○○○○○○○○○○○○○○○○○○○○○○○ ○○○○○○○○○○○○○○○○○○○○○○○○○○○○○○○○○○○○○○○○○○○○○○○○○○○○○○○○○○○○○○○○○○○○○○○○○○○○○○○○○○○○○○○○○○○○○○○○○○○○○○○○
概要には目的,背景,手法,実験方法,実験結果を簡潔にまとめたものを書く(つまり,論文を1ページに圧縮したものである.概要だけを読んでも,大まかに何をやって,どのような結果になったかが分かるように書くこと.0.5ページ分くらいはうめること.).○○○○ ○○○○○○○○○○ ○○○○○○○○○○○○○○○○○○○○○○○○○○○○○○○○○○○○○○○○○○○○○○○○○○○○○○○○ ○○○○○○○○○○○○○○○○○○○○○○○○○○○○○○○○○○○○○○○○○○○○○○○○○○○○○○○○ ○○○○○○○○○○○○○○○○○○○○○○○○○○○○○○○○○○○○○○○○○○○○○○○○○○○○○○○○ ○○○○○○○○○○○○○○○○○○○○○○○○○○○○○○○○○○○○○○○○○○○○○○○○○○○○○○○○ ○○○○○○○○○○○○○○ ○○○○○○○○○○○○○○○○○○○○○○○○○○○○○○○○○○○○○○○○○○○○○○○○○○○○○○○○ ○○○○○○○○○○○○○○○○○○○○○○○○○○○○○○○○○○○○○○○○○○○○○○○○○○○○○○○○ ○○○○○○○○○○○○○○○○○○○○○○○○○○○○○○○○○○○○○○○○○○○○○○○○○○○○○○○○ ○○○○○○○○○○○○○○○○○○○○○○○○○○○○○○○○○○○○○○○○○○○○○○○○○○○○○○○○○○○○○○○○○○○○○○○○○○○○ ○○○○○○○○○○○○○○○○○○○○○○○○○○○○○○○○○○○○○○○○○○○○○○○○○○○○○○○○○○○○○○○○○○○○○○○○○○○○○○ ○○○○○○○○○○○○○○○○○○○○○○○○○○○○○○○○○○○○○○○○○○○○○○○○○○○○○○○○○○○○○○○○○○○○○○○○○○○○○○○○○○○○○○○○○○○○○○○○○○

%%%%%%%%%%%%%%%%%%%%%%%%%%%%%%%%% 概要ここまで
\clearpage
