% TEX STUDIO MAGIC-COMMAND
% !TeX document-id = {21ffa6e2-6c8f-4532-897c-386dc477f19a}
% !TeX root = presen.tex
% !TeX encoding = utf8
% !TeX TXS-program:compile = lualatex -synctex=1 -interaction=nonstopmode -halt-on-error %.tex
% !TeX TXS-program:quick = txs:///compile | txs:///view-pdf-internal --embedded
%%%-------------------------------------------------------------------------
%%% PD3プレゼンプレート
%%% 作成: 金沢工大・情報工学科・鷹合研究室
%%%-------------------------------------------------------------------------

\input{tkg_slide.tex}

%%%%%%%%%%%%%%%%%%%%%%%%%%%%%%%%%%%%%%%%%
\renewcommand{\lstlistingname}{リスト}

% 図・表・リストのcaption番号を表示するか/表示しないかを選ぶ
\iffalse
\usepackage[hang,bf,labelformat = empty,labelsep=none,figurename=Y, tablename=X, singlelinecheck=off,justification=centering,labelfont=bf,textfont=bf]{caption} 
\else
\usepackage[hang,bf,labelsep=colon,figurename=図, tablename=表, singlelinecheck=off,justification=centering,labelfont=bf,textfont=bf]{caption} 
\fi

%%%%%%%%%%%%%%%%%%%%%%%%%%%%%%%%%%%%%%%%%
% 
% タイトルスライドのロゴ画像
% フッタ(左)
%%%%%%%%%%%%%%%%%%%%%%%%%%%%%%%%%%%%%%%%%
%  フッタ(左側)

  \MyLogo{\includegraphics[height=1.1cm]{fig/logo/kit_landscape1.pdf}}
% \MyLogo{--- 鷹合研究室 ---} % トップスライドの下部中央

  \lfoot{\includegraphics[height=.75cm]{fig/logo/kit_landscape1.pdf}}
% \lfoot{\small 鷹合研}        % フッタ(左)

%%%%%%%%%%%%%%%%%%%%%%%%%%%%%%%%%%%%%%%%%
% 
% フッタ(中央,右)
%
%%%%%%%%%%%%%%%%%%%%%%%%%%%%%%%%%%%%%%%%%
%\cfoot{\thepage/\pageref{LastPage}} 
\cfoot{\thepage/\pageref{LastPage}}
\rfoot{\small 4EP404} % テーマ番号

%%%%%%%%%%%%%%%%%%%%%%%%%%%%%%%%%%%%%%%%%%%
% ページ番号を1からにしたら,トップスライドの下部のロゴがうまくいかなくなったのでこうしてみた
\fancypagestyle{myfirstpage}
{
  \fancyhf{}
   \fancyfoot[C]{\includegraphics[height=1.1cm]{fig/logo/kit_landscape1.pdf}}
%  \fancyfoot[C]{鷹合研究室}
   \renewcommand{\headrulewidth}{0pt} % removes horizontal header line
}
%%


%%%%%%%%%%%%%%%%%%%%%%%%%%%%%%%%%%%%%%%%%
% 
% ここから下を書き換えて下さい 
%
%%%%%%%%%%%%%%%%%%%%%%%%%%%%%%%%%%%%%%%%%

\title{
{\normalsize 令和98年度 プロジェクトデザインIII}\\\vspace{10mm}
{\LARGE WebAssemblyを用いた\\グラフィクスレンダリングの高速化}
}
\date{令和99年99月99日}
\author{
4EP4-04\\ \ruby{天羽}{あもう}\ruby{大樹}{たいき}
}



\begin{document}
\maketitle % タイトルページ
\addtocounter{page}{1}
\thispagestyle{myfirstpage}

%%%%%%%%%%%%%%%%%%%%%%%%%%%%%
 \foilhead{\Large 1. はじめに -- 背景と目的 -- }
\begin{itemize}
 \item 将来、Webアプリケーションとして、リアルタイムに変化するxRコンテンツを作成し、動作させる際、グラフィクスレンダリングの速度が遅くなってしまうと、没入感を損なってしまう問題があると考えられる
 \item そのような問題に対処するために、Web上でグラフィクスのレンダリングを早く処理するために、どのような手法を用いれば、より早く処理できるのかという取り組みを行う必要がある
 \item 本プロジェクトにおいては、WebAssemblyという仕組みを用いたバイナリフォーマットプログラムを使って、グラフィクスレンダリングを早く処理させるプログラムを作成する
\end{itemize}
\newpage

%%%%%%%%%%%%%%%%%%%%%%%%%%%%%
\foilhead{\Large 発表の流れ}
\begin{enumerate}[itemsep=0.25\zh]
	\item \textcolor{gray}{はじめに -- 背景と目的 --}
	\item \textcolor{red}{用語などの簡潔な説明}
	\item プログラム概要
	\item 性能比較評価・考察
	\item むすび
\end{enumerate}
\newpage

%%%%%%%%%%%%%%%%%%%%%%%%%%%%%%%%%%%%%%%%%%%%%%
\foilhead{\Large 2. 用語の簡潔な説明}
\begin{itemize}
	\item WebAssembly\\Web上で動作するバイナリフォーマットプログラムのこと。\\wasmと略して呼称されることが多いため、本スライドでも\\その略称を以後使用する。特定のプログラミング言語で記述した\\プログラムをそのバイナリフォーマットにコンパイルしたものを\\主に指す。そのプログラムを保存したファイルは、\\wasmモジュールと呼称されることが多い。
	\newpage
	
	\item WebGL\\Web上でグラフィクスを扱うために、ブラウザに組み込まれているAPI\\後述するWebGPUより歴史が古く関連ライブラリが豊富で、\\Three.jsやBabylon.jsといったライブラリが有名。
	
	\item WebGPU\\WebGLより高度にGPUの性能を活かし、高速なグラフィクス\\レンダリングを行うことが可能であるとされている\\ブラウザ組み込みのAPI。まだ公開されてから日が浅いため、\\著名な関連ライブラリなどはまだあまり見受けられない
\end{itemize}
\newpage
%%%%%%%%%%%%%%%%%%%%%%%%%%%%%
\foilhead{\Large 発表の流れ}
\begin{enumerate}[itemsep=0.25\zh]
	\item \textcolor{gray}{はじめに -- 背景と目的 --}
	\item \textcolor{gray}{用語などの簡潔な説明}
	\item \textcolor{red}{プログラム概要}
	\item 性能比較評価・考察
	\item むすび
\end{enumerate}
\newpage

%%%%%%%%%%%%%%%%%%%%%%%%%%%%%%%%%%%%%%%%%%%%%%

\foilhead{\Large 3. プログラム概要}
Rustでブラウザに搭載されているAPIにアクセスするプログラムを記述し、\\そのプログラムをWebAssemblyにコンパイルする。\\コンパイルされたプログラムをJavaScriptでロードし、実行させる。\\
WebAssembly側で処理するプログラムの対象として、
\begin{itemize}
	\item 比較的、ハードウェア側に近いプログラムの処理
	\item 行列などを扱う演算負荷の高い処理
\end{itemize}
というような処理を指定して行う

%% \begin{figure}[h]
%% \begin{center}
%% \includegraphics[width=\textwidth]{fig/system.pdf}
%% \caption{}
%% \end{center}
%% \end{figure}
\newpage

%%%%%%%%%%%%%%% minipage の利用例 %%%%%%%%%%%%%%%%%%%
%------ 左側
\begin{minipage}[t]{0.4\textwidth}\vspace{0pt}
本プロジェクトでは、2つの\\プログラミング言語を用いて\\高速化を行う

\begin{enumerate}[parsep=-0.5\zh]
	\item Rustのプログラムで\\グラフィクスを操作するAPIをコール
	\item Rustのプログラムを\\WebAssemblyに\\コンパイルする
	\item WebAssemblyプログラムをJavaScript側から\\コールする
\end{enumerate}
\end{minipage}
%------ 右側
\begin{minipage}[t]{0.6\textwidth}\vspace{0pt}
\begin{center}
\includegraphics[keepaspectratio, width=.9\linewidth,trim={0mm 0mm 0mm 0mm},clip]{fig/system.pdf}
\end{center}
\end{minipage}

%%%%%%%%%%%%%%%%%%%%%%%%%%%%%%%%%%%%%%%%%%%%%%
\foilhead{\Large 3-1. 作成したWebAssemblyプログラムの概要}
\begin{description}

	\item[処理内容]~\\
	WebGPU APIを通してレンダリング内容を決め、実行する内容をバッファデータとして処理し、順番に実行させる一連の処理を行う
	\item[実行される処理の結果]~\\
	HTMLのcanvas要素の範囲にプログラムで指定した\\矩形や図形がレンダリングされる

\end{description}
\newpage

%%%%%%%%%%%%%%%%%%%%%%%%%%%%%
\foilhead{\Large 発表の流れ}
\begin{enumerate}[itemsep=0.25\zh]
	\item \textcolor{gray}{はじめに -- 背景と目的 --}
	\item \textcolor{gray}{用語などの簡潔な説明}
	\item \textcolor{gray}{プログラム概要}
	\item \textcolor{red}{性能比較評価・考察}
	\item むすび
\end{enumerate}
\newpage

%%%%%%%%%%%%%%%%%%%%%%%%%%%%%%%%%%%%%%%%%%%%%%
\foilhead{\Large 4. 性能比較評価・考察}

\begin{description}
	
	\item[評価方法]~\\
	JavaScriptのみで組んだ関数と、WebAssemblyを交えたプログラムを同時に実行して、その2つの関数の実行が終了するまでの時間を計測し、比較を行う
	\newpage
	\item[評価結果]~\\
	今回の比較では、以下の結果が出た。このケースではそれほど大きな差が出なかった。
	\item[考察]~\\
	今回は単純な単色の図形、もしくは三色のグラデーションの図形のみの描画であるため、大きな差にはならなかったと考えられる。しかし、2つの比較で、それぞれ速度差が現れているため、より内部の数値処理を増やすと、より大きな差が生まれる可能性があると考えられる
	
\end{description}

% 	\item このスライドでは何をどのような方法で評価したかを明記し,結果をグラフで示すこと(表よりグラフのほうが良い).
%	\item システムが動いている様子がわかるようにデモ映像を流すこと(デモ映像には字幕をつけたりするなどしてわかりやすくすること).
%	\item 評価の際は,改良の前後でどうなったかを示す.あるいは他の手法などと比較してどうなのかを示すことも必要.
%	\item 結果について考察も示すこと.

\newpage

%%%%%%%%%%%%%%%%%%%%%%%%%%%%%%%%%%%%%%%%%%%%%%
\foilhead{\Large 5. むすび}\label{MUSUBI}
\begin{itemize}
	\item Web上でのグラフィクスレンダリングを高速化するため、WebAssemblyを用いたモジュールを作成した。
	\item 現時点では、それほど大きな描画速度の差がないが、計算する数値が増えれば増えるほど、差が開いていく可能性があると考えられる。
	\item 来月の報告までに、より計算量が多くなるプログラムで比較を行う
\end{itemize}
\newpage

%%%%%%%%%%%%%%%%%%%%%%%%%%%%%%%%%%%%%
ここからおまけ

\href{run:./demo002.mp4}{\textcolor[hsb]{0.0, 0.7, 1.0}{\faPlayCircle[regular]}} PDFファイルと同じフォルダにdemo002.mp4があれば再生できる.


\href{https://youtu.be/74agBeJxdFI}{\textcolor{red}{\faYoutube}} YOUTUBEで再生

\textcolor{red}{\faYoutube}\href{https://youtu.be/74agBeJxdFI}{~\url{https://youtu.be/74agBeJxdFI}}

\lstinputlisting[language=c, caption=test2.c]{src/hello.c}
\lstinputlisting[language=python, caption=test2.py]{src/world.py}

% 色定義
\definecolor{mygray}{gray}{0.95}
\definecolor{mypink1}{hsb}{0.0, 0.188, 1.0}
UNIXv1におけるタスク切り替えが行われるタイミング

%%%%%%%%%%%%%%%%%%%%%%%%55
\colorbox{mygray}{\begin{minipage}{\textwidth}
① みなさん
\end{minipage}}

\colorbox{mygray}{\begin{minipage}{\textwidth}
② こんにちは 
\begin{itemize}
\item まんじゅう
\item りんご
\end{itemize}
\end{minipage}}

\colorbox{mypink1}{\begin{minipage}{\textwidth}
③ お元気で\\
またあうひまで
\end{minipage}}
%%%%%%%%%%%%%%%%%%%%%%%%%%%%%%%%%%%%%%%%%%%%

\begin{verbatimx}
$ gcc test.c \return
 (*_*)
 (*_*)
        \textcolor{red}{ここで\keytop{CTL}+\keytop{C}を押す}
\end{verbatimx}
%%%%%%%%%%%%%%%%%%%%%%%%%%%%%%%%%%%%%%%%%%%%%%%%%%%%%%%%%%%%%%%%%%%%%%%%
\newpage
~\\
\noindent\textbf{謝辞}~~本研究はJSPS科研費21Kxxxxxxxxx助成を受けた
%%%%%%%%%%%%%%%%%%%%%%%%%%%%%% 参考文献 %%%%%%%%%%%%%%%%%%%%%%%%%%%%%%
\begin{thebibliography}{99}
\small
\setlength\itemsep{-0.5\zh}%
\bibitem{book1} K.Thompson,D.M.Ritchie,\textbf{"The UNIX Time-Sharing System"},Communications of the ACM, Vol.17, No.7, 1974.
\bibitem{book4} Digital Equipment Corporation: \textbf{PDP11/20-15-r20 Processor Handbook}, 1971.
\bibitem{Preliminary} T.R. Bashkow, \textbf{"Study of UNIX: Preliminary Release of Unix Implementation Document"}, \url{ http://minnie.tuhs.org/Archive/Distributions/Research/Dennis_v1/PreliminaryUnixImplementationDocument_Jun72.pdf}, Jun. 1972.
%\bibitem{book2} K. Thompson,D.M. Ritchie,"UNIX PROGRAMER'S MANUAL",Nov. 1971.
%\bibitem{web0} Warren Toomey, "The Unix Heritage Society", \url{https://www.tuhs.org/}, Dec. 2015.
\bibitem{simh} simh, \textbf{"The Computer History Simulation Project"}, \url{https://github.com/simh/simh}, 参照Mar.14, 2022.
\bibitem{ref0} W.Toomey, \textbf{"First Edition Unix: Its Creation and Restoration"}, IEEE Annals of the History of Computing, 32 (3), pp.74-82, 2010.
%\bibitem{web1} Jim Huang, "Restoration of 1st Edition UNIX from Bell Laboratories", \url{https://github.com/jserv/unix-v1}, 参照Mar.14, 2022.
\bibitem{book3} Diomidis.Spinellis,\textbf{"unix-history-repo"},  \url{https://github.com/dspinellis/unix-history-repo/tree/Research-V1}, 参照Mar.14, 2022.
\bibitem{book5} Digital Equipment Copporation: \textbf{PDP11 Peripherals HandBook}, 1972.
%\bibitem{book6} \url{https://github.com/No000/unix-v1-utils}
%\bibitem{book7} \url{https://github.com/No000/UnixV1-SystemCallTracer}
\end{thebibliography}

\end{document} 
